\setcounter{secnumdepth}{4}

\usepackage[explicit]{titlesec}

\titleformat
    {\chapter}
    {\bfseries\centering}
    {\thechapter.}{1em}{\MakeUppercase{#1}}

\titlespacing*
    {\chapter}
    {0em}
    {0.7em minus \baselineskip}
    {0.7em}

\titleformat
    {\section}
    {\bfseries}
    {\thesection.}{1em}{#1}

\titlespacing*
    {\section}
    {\parindent}
    {0.7em}
    {0.7em}

\titleformat
    {\subsection}
    {\bfseries}
    {\thesubsection.}{1em}{#1}

\titlespacing*
    {\subsection}
    {\parindent}
    {0.7em}
    {0.7em}

\titleformat
    {\subsubsection}
    {\bfseries}
    {\thesubsubsection.}{1em}{#1}

\titlespacing*
    {\subsubsection}
    {\parindent}
    {0.7em}
    {0.7em}

\titleformat
    {\paragraph}
    {\bfseries}
    {\theparagraph.}{1em}{#1}

\titlespacing*
    {\paragraph}
    {\parindent}
    {0.7em}
    {0.7em}

\titleformat
    {\subparagraph}
    {\bfseries}
    {\thesubparagraph.}{1em}{#1}

\titlespacing*
    {\subparagraph}
    {\parindent}
    {0.7em}
    {0.7em}

% Links break after `\MakeUppercase`, see:
% https://github.com/latex3/hyperref/issues/121.
% The good thing is there is an older command
% `\uppercase` which does the same, but without
% expanding the argument (which is ok since we
% use a utf-8 engine).

\usepackage{titletoc}

\newcommand\chapterlabelindent{1.4em}
\newcommand\sectionlabelindent{2.3em}
\newcommand\subsectionlabelindent{3.2em}

\titlecontents{chapter}[\chapterlabelindent]
    {\addvspace{0.15em}}
    {\bfseries\contentslabel[\thecontentslabel.]{\chapterlabelindent}\uppercase}
    {\hspace*{-\chapterlabelindent}\bfseries\uppercase}
    {\titlerule*[1pc]{.}\bfseries\contentspage}
    [\addvspace{0.15em}]

\titlecontents{section}[\sectionlabelindent]
    {}
    {\contentslabel[\thecontentslabel.]{\sectionlabelindent}}
    {\hspace*{-\sectionlabelindent}}
    {\titlerule*[1pc]{.}\contentspage}
    [\addvspace{0em}]

\titlecontents{figure}[\sectionlabelindent]
    {}
    {\contentslabel[\thecontentslabel.]{\sectionlabelindent}}
    {\hspace*{-\sectionlabelindent}}
    {\titlerule*[1pc]{.}\contentspage}
    [\addvspace{0em}]

\titlecontents{table}[\sectionlabelindent]
    {}
    {\contentslabel[\thecontentslabel.]{\sectionlabelindent}}
    {\hspace*{-\sectionlabelindent}}
    {\titlerule*[1pc]{.}\contentspage}
    [\addvspace{0em}]

\titlecontents{subsection}[\subsectionlabelindent]
    {}
    {\contentslabel[\thecontentslabel.]{\subsectionlabelindent}}
    {\hspace*{-\subsectionlabelindent}}
    {\titlerule*[1pc]{.}\contentspage}
    [\addvspace{0em}]

% The following code is a fix that makes
% minted lists of listings be styled the same
% as the list of figures. This is needed,
% because titletoc can't directly style `minted`s
% list of listings.
% https://tex.stackexchange.com/questions/58469/why-are-listof-and-listoffigures-styled-differently

\usepackage{etoolbox}

\makeatletter
    \patchcmd{\@chapter}
        {\addtocontents{lof}}
        {
            \addtocontents{lol}{\protect\addvspace{10pt}}
            \addtocontents{lof}
        }
        {\typeout{*** SUCCESS ***}}
        {\typeout{*** FAIL ***}}

    \renewcommand*{\listof}[2]{%
        \@ifundefined{ext@#1}{\float@error{#1}}{
            \expandafter\let\csname l@#1\endcsname \l@figure % <- use layout of figure
            \float@listhead{#2}
            \begingroup
            % \setlength\parskip{0pt plus 1pt} % <- or drop this line completely
            \@starttoc{\@nameuse{ext@#1}}
            \endgroup
        }
    }
\makeatother
